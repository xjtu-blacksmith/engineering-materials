\documentclass[12pt,a4paper]{article}
\usepackage[margin=1in]{geometry}

\usepackage[heading]{ctex}
\usepackage{amsmath}
\usepackage{unicode-math}
\setCJKmainfont{思源宋体}
\setCJKsansfont{思源黑体}
\setmainfont{Cambria}
\setsansfont{Calibri}
\setmathfont{Cambria Math}

\usepackage{fancyhdr}
\usepackage{xcolor}
\lhead{}\rhead{}
\chead{\textsc{Outline on Fundamentals of Engineering Materials}}
\pagestyle{fancy}
\setcounter{secnumdepth}{0}
\setlength{\headheight}{15pt}

\usepackage{siunitx}
\usepackage{fontawesome5}

\newcommand{\tightlist}{\setlength{\parskip}{0pt}\setlength{\itemsep}{0pt}}
\newcommand{\hint}[1]{\textsf{(#1)}}
\newcommand{\minor}[1]{{\color{gray} #1}}
\newcommand{\then}{$\to$}
\newcommand{\equv}{$\Leftrightarrow$}
\newcommand{\beginday}{2019 年 9 月 15 日}
\renewcommand{\emph}[1]{\faIcon{lightbulb}\ \textbf{#1}}
\newcommand{\emphitem}[1]{\item[\faLightbulb]\ \textbf{#1}}

\newcommand{\ve}{\varepsilon}
\DeclareMathOperator{\inv}{inv}

\title{《工程材料基础》考点提纲}
\date{\beginday\ -- \today}
\author{黑山雁}

\begin{document}
\maketitle

\section{绪论}
\begin{itemize}\tightlist
    \item 材料决定着人类科技的进步水平:石器时代\to 青铜器时代\to 铁器时代\to 信息材料
    时代
    \item 材料两大作用方向:社会生产与发展的物质基础;实现特定功能。
    \item 工程材料学:以用于工程结构、机器零件的材料\hint{工程材料}为研究对象。
    \item 四大类:
    \begin{description}\tightlist
        \item[金属材料] 各项指标优异,易加工、处理,应用最广;
        \item[高分子材料] 高弹性、高耐磨性、绝缘好、成本低,强度低,用于低载荷场合;
        \item[陶瓷材料] 高硬度、绝缘好、耐高温腐蚀,韧性差、成本高,用于抗压力、绝缘等
        场合。
        \item[复合材料] 结合两种或两种以上材料特性,比强度、比刚度高,用于航空航天。 
    \end{description}
    \item 材料的性能:使用性能、工艺性能。
    \item 材料的使用性能:材料固有性能\hint{天赋};成分/组织结构\hint{家庭环境};合成、
    加工、处理工艺\hint{教育背景}。
    \emphitem{没有十全十美的材料;需要根据实际情况,平衡对材料的各项需求。}
\end{itemize}

\section{机械零件(或器件)的失效分析}

\begin{itemize}\tightlist
    \item 失效:失去设计要求的效能。
    \item 常见失效形式:过量变形、断裂、磨损、腐蚀。
\end{itemize}

\subsection{过量变形}

\begin{itemize}\tightlist
    \item 弹性变形与塑性变形。
    \item 常见材料的应力 -- 应变曲线走势。
    \item 静载性能指标:刚度 $EA$ \hint{抵抗弹性变形}、强度 $\sigma$ \hint{抵抗变形
    与断裂}。\emph{刚度一定是与弹性变形有关,不涉及塑性变形!}
    \item 刚度与强度分别用弹性能 $u=\frac12\sigma_e\ve_e$、断后伸长率 $A$ 与断面收缩
    率 $Z$ 作为评价指标。
    \item 刚度指标\hint{弹性模量 $E$}不易由加工改变,强度指标\hint{各强度极限}容易由
    加工改变!
\end{itemize}

\subsection{断裂}

\begin{itemize}\tightlist
    \item 断裂:材料在应力作用下分为两个或以上部分。
    \begin{description}\tightlist
        \item[韧性断裂] 断裂前发生明显宏观塑性变形\hint{低碳钢拉伸}。
        \item[脆性断裂] 断裂前不发生塑性变形\hint{灰铸铁拉伸}。
    \end{description}
    \emphitem{脆性断裂更为致命,需要警惕。}
    \item 断裂过程:裂纹形成\to 扩展\hint{亚稳扩展、失稳扩展};裂纹的临界长度 $a_c$:
    亚稳与失稳的分界线。
    \item 脆性断裂:$a_c$ 太低。
    \emphitem{韧性是材料强度与塑形的综合体现。}
    \item 冲击韧性:材料在冲击载荷下吸收塑性变形功、断裂功的能力,用冲击韧度 $a_K=A_K/
    F_K$\hint{冲击功比截面积}评价。
    \minor{\item 常用冲击试验确定材料的冲击功,进而计算冲击韧性。}
    \item 韧脆转变温度 $T_K$:在\hint{低}温下材料韧性急剧下降的一温度点;低温脆性断裂
    现象。
    \item 冲击韧性指标的不足:未考虑到材料内部缺陷\hint{微小的宏观裂纹}。
    \item 断裂韧性:材料抵抗脆性断裂的能力,用
    \[K_{Ic}=Y\sigma\sqrt a\]
    校核\hint{$Y=1\sim2$ 为几何形状因子,$\sigma$ 为零件工作应力,$a$ 为裂纹长度。}
    \item 断裂韧度 $K_{Ic}=K_I(a_c)$;$K_I<K_{Ic}$ 为校核条件。
    \item \minor{三类断裂形式:拉伸断裂(I)、剪切断裂(II),扭转断裂(III)。}
    \item 断裂韧度比较:纯金属 $>$ 合金 $>$ 复合材料 $>$ 高分子、陶瓷。
    \item 加载方式与材料属性对断裂的影响:作 $\tau_{\max}$ -- $\sigma_{\max}$ 图分析。
    % 此处应列子条目详述原则。
    \item 疲劳断裂:经长时间交变载荷作用而断裂。
\end{itemize}

\subsection{磨损}

\end{document}